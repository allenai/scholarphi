% Before processing this TeX file, you must load the LaTeX color commands.

% '@' needs to be processed in command names so we can access lower-level macros
% (e.g., '\current@color', '\Gin@driver') that will be defined by the TeX engines
% to let us know what's doing the coloring, and to let us set colors.
\catcode`@ = 11

% Define names of color drivers
% You can check the name of the current driver using '\Gin@driver{}'
\def\pdftexdrivername{pdftex.def}
\def\dvipsdrivername{dvips.def}

\scholarifdefined{current@color}{%
% scholarsetcolor: Set color for everything in the document after this command.
% Color will apply even after the current group is finished.
\def\scholarsetcolor[#1]#2{%
{\csname color@#1\endcsname\current@color{#2}%
\ifx\Gin@driver\pdftexdrivername%
\pdfcolorstack0 push {\current@color}%
\else\ifx\Gin@driver\dvipsdrivername%
\special{color push \current@color}%
\else\message{Coloring not implemented for driver \Gin@driver}%
\fi\fi%
}%
}%
% scholarrevertcolor: Revert a color set in a 'scholarsetcolor' command.
\def\scholarrevertcolor{%
\ifx\Gin@driver\pdftexdrivername%
\pdfcolorstack0 pop%
\else\ifx\Gin@driver\dvipsdrivername%
\special{color pop}%
\else\message{Coloring not implemented for driver \Gin@driver}%
\fi\fi%
}%
\message{Defined S2 LaTeX coloring commands.}%
}%

% scholarregistercitecolor: Register a color for a citation key. Everywhere a
% source is cited using this key, the citation body will appear in this color.
% Takes four arguments: citation key, red, green, blue.
\def\scholarregistercitecolor#1#2#3#4{%
\expandafter\def\csname scholarcolor@#1\endcsname{#2,#3,#4}%
\definecolor{scholarcolor@#1}{rgb}{#2,#3,#4}
}

% Define rules for coloring citations.
% First, instrument default LaTeX citation formatting command to insert a color
% command when its formatting a known citation.
\let\scholar@inner@cite@ofmt\@cite@ofmt
\def\@cite@ofmt#1{%
\scholarifdefinedelse{scholarcolor@\@citeb}{%
\textcolor{scholarcolor@\@citeb}{\scholar@inner@cite@ofmt{#1}}%
}{%
\scholar@inner@cite@ofmt{#1}%
}}

% Second, instrument hyperref citation coloring commands to use the colors
% specified by 'scholarregistercitecolor' instead of defaults.
\let\scholar@inner@citecolor\@citecolor
\def\@citecolor{%
\scholarifdefinedelse{scholarcolor@\@citeb}{%
scholarcolor@\@citeb%
}{%
\scholar@inner@citecolor%
}}

% Revert '@' to just be a normal character.
\catcode`@ = 12