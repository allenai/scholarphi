% Before processing this TeX file, you must load the LaTeX color commands.

% '@' needs to be processed in command names so we can access lower-level macros
% (e.g., '\current@color', '\Gin@driver') that will be defined by the TeX engines
% to let us know what's doing the coloring, and to let us set colors.
\catcode`@ = 11

% Define names of color drivers
% You can check the name of the current driver using '\Gin@driver{}'
\def\pdftexdrivername{pdftex.def}
\def\dvipsdrivername{dvips.def}

\scholarifdefined{current@color}{%
% scholarset color: Set color for everything in the document after this command.
% Color will apply even after the current group is finished.
\def\scholarsetcolor[#1]#2{%
{\csname color@#1\endcsname\current@color{#2}%
\ifx\Gin@driver\pdftexdrivername%
\pdfcolorstack0 push {\current@color}%
\else\ifx\Gin@driver\dvipsdrivername%
\special{color push \current@color}%
\else\message{Coloring not implemented for driver \Gin@driver}%
\fi\fi%
}%
}%
% Command for reverting a color set in an 'scholarsetcolor' command.
\def\scholarrevertcolor{%
\ifx\Gin@driver\pdftexdrivername%
\pdfcolorstack0 pop%
\else\ifx\Gin@driver\dvipsdrivername%
\special{color pop}%
\else\message{Coloring not implemented for driver \Gin@driver}%
\fi\fi%
}%
\message{Defined S2 LaTeX coloring commands.}
}%

% Revert '@' to just be a normal character.
\catcode`@ = 12